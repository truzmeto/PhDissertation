%\documentclass[../Dissertation.tex]{subfiles}
\documentclass[../talant.diss.submit.tex]{subfiles}
\begin{document}
\label{chap:chapter3}
%
% --------------------------------------------------------------------------------------------------
\section{\textbf{Introduction}}\label{sect:three_one}
%\section{Introduction}
%########################### History of IDPs ########################################################
Nearly $20\%$ of proteins in eukaryotic cell are found unfolded or at least have unstructured domains
under physiological conditions. Such "intrinsically disordered" proteins (IDPs) are void of well defined
secondary or tertiary structure in a native state, possessing instead ensemble of rapidly changing conformations
\cite{wright:99,dunker:01}. Although natively unfolded in a solution, IDPs usually
(though not always\cite{sigalov:07bin,chakrabortee:10cat})
fold upon association to target receptors through so called coupled folding and binding mechanism
\cite{wright:99,sugase:07,huang:09,turjanski:08}, where IDP undergoes structural formation assisted by
intermolecular interactions with binding partner.
%Two extreme scenario's for coupled folding and binding
%are confirmations selection (folding prior to binding) and induced fit (folding upon/after binding) baseline
%mechanisms have been postulated.\cite{baker:14,espinoza-fonseca:09a}
%
%Experimental
%evidence for induced fit mechanism was demonstrated by Sugase et al for pKID and KIX binding through
%relaxation and dispersion NMR spectroscopy, where they observed seconary structure formation of pKID
%helicies as it binds to KIX\cite{sugase:07}.
%
%######################## Advantages over ordered protein's
%Physical properties inherent in IDPs give rise to functional advantages over the folded proteins such
%as (i) kinetic advantege \cite{shoeportman:00}, (ii) low affinity with high specificity
%binding\cite{uversky:05}, (iii) enlarged interaction surface\cite{gunasekaran:03}, (iv) ability
%to interact with multiple targets due to conformational flexibility\cite{kriwacki:96}.

%############################### Fly-casting ########################################################
Abundance of IDPs in a cell makes coupled folding and binding a common and vital bimolecular activity.
Majority of functions they carry mostly involve regulation or signal transduction. \cite{gsponer:09r}
One particular property that received wide attention lately is their kinetic advantage over
folded proteins facilitated by flexible and dynamic nature inherent in IDPs. This rapid molecular recognition
associated with coupled folding and binding was originally proposed by Shoemaker $\textit{et~al.}$
via so called fly-casting mechanism\cite{shoeportman:00}. Proposed framework can be explained in two consecutive
steps: First, due to lack of structure in the unbound state,IDPs are large in size, which enables weak and
non-specific interactions with the target molecule at farther distances, therefore enhancing the capture rate.
Second, intermolecular interactions encourage folding near binding site, thus resulting in fast
activation (barrier crossing) rate.
Although IDPs tend to have faster binding rates than folded proteins, they do not always reach
the upper-limit of bi-molecular binding rates.\cite{rogers:13} In this chapter, via coarse grained native-centric
simulations, we explore how the picture described by the fly-casting mechanism carries over to the activated binding
regime. Rather than taking the viewpoint that deviations from diffusion-limited binding is evidence against a
fly-casting description\cite{huang:09}, we illustrate that the conceptual framework of fly-casting
persists even when the binding rates are activated.

Coarse grained structure based model simulations\cite{levy:05,okazaki:06} confirm fly-casting either via kinetic or
thermodynamic binding analysis at different ligand flexibilities.
\cite{turjanski:08,huang:09,levy:04}
Long range electrostatic interactions have been shown to increase the ``fly-casting`` effect of IDPs.\cite{levy:07}
Non-native hydrophobic interactions seem to accelerate binding rate initially, but dramatical reduction in
rate is observed as residues get trapped in a non-specific states at relatively strong non-native contact
energies\cite{huang:10a}.
%By conducting critical assessment on fly-casting,
%Huang $\it{et.al}$
%\cite{huang:09} suggested that larger geometric size in the unbound might be kinetic disadvantage, rather than
%advantage, based on slightly reduced capture rate. This is most likely due to comparing capture rates of flexible
%and rigid IDPs at large separation of melting temperatures. If proper calibration of intermolecular interactions
%is performed to bring melting temperatures close, while preserving the balance between bound and unbound states,
%then it can be shown that capture rate of IDP is faster than the capture rate of its ordered counterpart.
%Observed ~2.5 fold faster overall association rate for IDP over it's ordered counterpart lead to a conclusion that
%binding enhancement is governed by number of attempts made before pKID evolves into bound complex, which qualitatively
%agrees with our findings.
In another interesting investigation by Ref.~\cite{huang:09}, Huang and Lui expressed skepticism over
fly-casting to explain the faster kinetics of coupled folding and binding based in part on the observation
that the rates of encounters were not sensitive to protein flexibility (in fact, they found rigid proteins had
a slightly higher encounter rate). Enhanced binding rate due to coupled folding and binding is often explained
by appealing to a diffusion-limited picture of binding. Summarizing the physical basis for fast binding rates of
IDPs as an ``increased capture radius'' due to the expanded and dynamic conformations of an
unfolded protein\cite{shoeportman:00} may encourage thinking about coupled folding and binding
only in the diffusion-limited regime. For example, such comparisons evoke the effective size
of monomers due to the influence of electrostatic interactions on diffusion limited binding
rates. Here, we explore how basic conceptual framework of fly-casting carries over
to the case where overcoming a barrier dominates binding kinetics.
%
%Using topology based model, Levy $\it{et~al.}$
%\cite{levy:04} illustrated fly-casting on dimerization
%process of partially unfolded monomers that suggested the applicability of proposed framework to wide range of
%cellular species.
%

Reaction limited binding in an implicit solvent simulations can be introduced by raising the activation barrier through
potential that accounts for solvent expulsion upon reaction. So called desolvation barrier potential % Fig.~\ref{fig:potential}
has been shown to increase free energy barrier at expanse of displacing water molecules upon tight packing of solvent
exposed sites.\cite{levy:06,cheung:02,liu:05}
%Implicite solvent simulations usually employ such an effect
%by introducing the potential that mimics desolvation. For instance, tight packing between solvent exposed binding
%sites of proteins is achieved at expanse of solvent expulsion, thus increasing the free energy barrier. 
In addition, this potential has been shown to enhance folding cooperativity \cite{chan:11,liu:05a} as well as produce
folding rates that span a range consistent with experimental observation.\cite{ferguson:09}


%############################--Binding Affinity--#########################################
Another peculiar advantage of intrinsic disorder is having low affinity with high specificity
binding.\cite{uversky:05} Upon binding and folding IDPs loose conformational entropy. 
It's been suggested that this loss is not entropically favorable for binding, thereby uncoupling
binding affinity from specificity, thus allowing highly specific binding with low affinity.
\cite{dyson:05,zhou:12}. Binding affinity and rates are simply related through the following
expression: $K_{\mathrm{a}} = k_{\mathrm{on}}/k_{\mathrm{off}}$. Zhou argued that
having both high affinity $K_{\mathrm{a}}$ and dissociation rate constant $k_{\mathrm{off}}$ could
be problematic since association rate constant has an upper limit dictated by diffusion.\cite{zhou:12}
As an explanation, he suggested that IDP's overcome this problem by adjusting dissociation rate
$k_{\mathrm{off}}$.

%################### last secion
Our analysis of the simulated coupled folding and binding of pKID-KIX reveals
gradual accumulation of binding contacts and resolution of orientational constraints
starting at a distance on the order of the size of the unfolded protein as a key difference
in the binding of pKID and a version of pKID which is constrained to be folded throughout
the simulation. Furthermore, these simulations emphasize that an unfolded protein's ability
to navigate desolvation barriers incrementally is a primary mechanism promoting fast binding
as well as high dissociation rate constant with low affinity. 


%
%Extensive experimental studies have been done to clarify if IDPs have kinetic advantage over
%ordered proteins, and it is found that not all IDPs have this advantage, thus making it poorly understood.
%
%
%Structure based models have been extensiely used as an efficient and alternative model over
%heavy all atom simulations thus gaining popularity in recent couple decades. In this model each
%aminoacid is treated as a mass point and impilicite incorporation of the solvent increases
%computation efficiency by two to three orders of magnitude. Although, this models are computationally
%very inexpansive and have been providing decent results, its applicability have raised some concernes.
%For instance, implicit incorporation of solvent effect doesn't capture effects such as desolvation of
%hydrophobic core. Consequently, desolvation barrier potential was introduced to implement solvent
%effect\cite{cheung:02}. \\
%
%
%\ About pKID-KIX Complex
%Phosphorylated kinase-inducible domain(pKID) is the transcription factor cAMP response-element binding
%protein (CREB) and KIX is the KID-binding domain of the CREB-binding protein. One of the well experimentally
%caracterized IDP complexes Fig.~\ref{fig:vmd}.
%The IDP pKID folds into
%two helices upon binding to KIX; $\alpha_{\mathrm{A}}$ (residues 119-129) and $\alpha_{\mathrm{B}}$
%(residues 132-146) joined through flexible linker\cite{zor:02}. Recent NMR experiments show that
%$\alpha_{\mathrm{A}}$ is partialy ordered with  $~50\% $ helical population and $\alpha_{\mathrm{B}}$ is
%mostly  disordered having about $10-15\%$ residual structure formed\cite{radhakrishnan:98}.
%Large hydrophobic residues (Tyr 134, Ile 137 and Leu 138) which belong to helix $\alpha_{\mathrm{B}}$ found
%to be essential for high-affinity binding \cite{radhakrishnan:97,parker:98}.
%
%
%
%
%######################################     Summary       ############################################  
%In this paper, we explain origins of kinetic advantage for IDP pKID over its ingeneered folded clone 
%as well as its ability to have high specificity with low affinity using native centric C-$\alpha$ model.

%By comparing binding affinities at the same temperature
%but distinct stabilities we demonstrate that difference in the off rate is the main reason
%for flexible binding partners to have such a high dissosiation constant over its ordered counterpart.
%Furthermore, thermodynmics analysis show that such a high off rate of IDP's is a result of its ability
%to overcome barriers incrementally upon dissociation.
%In a recent experimental investigation, by destabilizing helical structure of
%PUMA with proline mutation into surface residues, affinity for Mcl-1 was reduced.
%In addition, it was demostrated that increase in the off rate was mainly
%responsible for above decrease in binding affinity\cite{rogers:14}.\\


%*************************************************************************************************
%\section{\textbf{pKID-KIX IDP Complex}}\label{sect:three_two}
%*************************************************************************************************
%
%Phosphorylated kinase-inducible domain(pKID) is the transcription factor cAMP response-element binding
%protein (CREB) and KIX is the KID-binding domain of the CREB-binding protein. One of the well experimentally
%caracterized IDP complexes Fig.~\ref{fig:figure1}. The IDP pKID folds into
%two helices upon binding to KIX; $\alpha_{\mathrm{A}}$ (residues 119-129) and $\alpha_{\mathrm{B}}$
%(residues 132-146) joined through flexible linker\cite{zor:02}. Recent NMR experiments show that
%$\alpha_{\mathrm{A}}$ is partialy ordered with  $~50\% $ helical population and $\alpha_{\mathrm{B}}$ is
%mostly  disordered having about $10-15\%$ residual structure formed\cite{radhakrishnan:98}.
%Large hydrophobic residues (Tyr 134, Ile 137 and Leu 138) which belong to helix $\alpha_{\mathrm{B}}$ found
%to be essential for high-affinity binding \cite{radhakrishnan:97,parker:98}.
%
%
\begin{figure}[htp!]
  \begin{centering}
    \includegraphics[width=8.0cm]{figures/chap3_figs/figure1.pdf}
    \caption{NMR structure of pKID-KIX native state(PDB id: 1KDX) is
      shown, where silver substrate is KIX, and cyan represents IDP
      pKID. Native state structure for above complex is portrayed using
      visual molecular dynamics. ~\cite{Humphrey:96}}
    \label{fig:figure1}
  \end{centering}
\end{figure}
%
%
%*************************************************************************************************
\section{\textbf{Binding Thermodynamics}}\label{sect:three_twoo}
%*************************************************************************************************
%
%
We first characterize the thermodynamics of coupled folding and binding through
the free energy parameterized by two global order parameters that monitor the
folding of pKID ($Q_\mathrm{intra}$) and binding between pKID and KIX
($Q_\mathrm{inter}$). Here, $Q_\mathrm{intra}$ is the fraction of native
contacts formed between residues within pKID, and $Q_{\mathrm{inter}}$ is the
fraction of native contacts formed between pKID and KIX. As shown in
Fig.~\ref{fig:figure2} (a), unbound and disordered pKID remains
largely unstructured before overcoming a free energy barrier of approximately
$2.6\mathrm{kcal/mol}$ at $Q_\mathrm{inter} \approx 0.2$, after which folding
and binding progressively develop until binding is complete.  This interplay
between inter and intramolecular contact formation is in harmony with induced
fit coupled folding and binding mechanism.  In comparison,
Fig.~\ref{fig:figure2}(b) shows that when pKID is constrained to be folded
before binding, the free energy barrier separating bound and unbound ensembles
is approximately 6.5 kcal/mol at $Q_\mathrm{inter}\approx 0.4$. This barrier is
significantly high with more structured intermolecular interface compared to
transition barrier of induced fit binding.

\begin{figure}[h!]
  \centering
  \includegraphics[width=6.0in]{figure2.pdf}
  \caption{Free energy contour plots of folding and binding of pKID to KIX is
    shown for flexible (a; $\alpha=0.05$) and rigid (b; $\alpha=1.0$).  Free
    energy units are in $\mathrm{kcal/mol}$ and sampled at binding/unbinding
    transition temperatures(Tm) with WHAM.}
  \label{fig:figure2}
\end{figure}

A more detailed description of the thermodynamic binding mechanism is obtained
through local structural order parameters.  The degree of intermolecular
structure for each residue along pKID sequence as a function of the overall
native order of the interface between pKID and KIX is illustrated in
Fig.~\ref{fig:figure3}.  Moving from the bottom to the top of the plot with a
cross section of fixed $Q_{\mathrm{inter}}$ shows local interfacial order that
develops in the binding transition from unbound to a fully bound complex. The
colors in the plot reflect local binding order parameters, $q_{i}$,
defined as the fraction of intermolecular native contacts involving individual
residue $i$.

\begin{figure}[h!]
  \centering
  \includegraphics[width=5.50in]{figure3.pdf}
  \caption{Fraction of native intermolecular contacts for each individual
    pKID residue shown as a function of global binding reaction coordinate $Q_{\mathrm{\mathrm{inter}}}$.
    Normalized density for each residue is represented by color.}
  \label{fig:figure3}
\end{figure}

As shown in Fig.~\ref{fig:figure3}, the binding interface initiates with
$\alpha_{\mathrm{B}}$ followed by the binding of $\alpha_{\mathrm{A}}$.  More
specifically, the binding transition progresses in steps characterized by bound
state of three distinct regions along pKID sequence denoted as
$\alpha_{\mathrm{B-CT}}$ (residues 140-144),$\alpha_{\mathrm{B-NT}}$(residues
133-139) and $\alpha_{\mathrm{A-CT}}$(residues 124-128) consistent with regions
highlighted in experimental measurement\cite{sugase:07} and previously reported
simulations.\cite{ganguly:11} Coupled folding and binding initiates with the
C-terminal region of $\alpha_{\mathrm{B}}$ ($\alpha_{\mathrm{B-CT}}$), even
though pKID remains largely unstructured. After $\alpha_{\mathrm{B-CT}}$ binds
completely, residues in the N-terminal region on $\alpha_{\mathrm{B-NT}}$
engages in binding at $Q_{\mathrm{inter}}$=0.50, with still no significant
change in the overall folding of pKID (see Fig.~\ref{fig:figure2}a ).
Completion of $\alpha_{\mathrm{B-NT}}$ binding at $Q_{\mathrm{inter}}$=0.75
results in about 10\% additional intramolecular contacts above the order found
in the unbound ensemble.  Finally, with pKID anchored to KIX by
$\alpha_{\mathrm{B}}$, the C-terminus region of $\alpha_{\mathrm{A}}$ starts to
bind, near global free energy minimum shown in Fig.~\ref{fig:figure2}
($Q_{\mathrm{inter}}$=0.75) followed by the binding of the n-terminus of
$\alpha_{\mathrm{A}}$.

%
%*************************************************************************************************
\section{\textbf{Binding Kinetics}}\label{sect:three_three}
%*************************************************************************************************
%
Simulated binding rates are calculated from the mean first passage times,
$\tau_\mathrm{on}$, from 200 independent binding reactions. As shown in
Table~\ref{tab:kinetics}, coupled folding and binding enhances binding rate,
$k_\mathrm{on} = 1/\tau_{\mathrm{on}}$, by a factor of 7 times the binding rate
when pKID is structured before binding.  In comparison, course grained
simulations pKID-KIX with models that do not include desolvation barriers report
a smaller speed up in the rate (about 2-3 times) when pKID is
unfolded\cite{turjanski:08,huang:09}.

\begin{table}[t]
  \centering
  \begin{tabular}{@{}llllllll@{}}
    &$k_\mathrm{on/off}$
    \footnote{Rates expressed in $10^{-6} [\Delta t^{-1}]$, and time units are in $10^{6} [\Delta t]$}
    & $\langle n_{\mathrm{enc}}\rangle$
    & $\tau_{\mathrm{ec}}$ 
    & $k_{\mathrm{D}^{+}}$
    & $k_{\mathrm{D}^{-}}$
    & $k_{\mathrm{A}}$\\
    \hline
    flexible & 0.38  & 30  &  0.021 & 16.0  &  49.5  &  1.61   \\
    rigid    & 0.05  & 225 &  0.018 & 14.3  &  56.8  &  0.25   \\
  \end{tabular}
  \caption{Binding kinetics simulations performed at $T_0=321^\circ$ K for a flexible, and $T_0=517^\circ$ K for
    the rigid pKID. Rates expressed in $10^{-6} [\Delta t^{-1}]$, and time units are in $10^{6} [\Delta t]$.}
  \label{tab:kinetics}
\end{table}


To understand this difference in more detail, we consider binding kinetics
that proceeds through a loosely bound encounter complex with
a subsequent activated step that leads to the final bound complex:\cite{sugase:07,huang:09}
\begin{equation}
  \label{eq:kinetic_equation}
  \ce{
    \mbox{pKID + KIX} <=>[$k_\mathrm{D}^{+}$][$k_\mathrm{D}^{-}$] \mbox{pKID:KIX}
    ->[$k_\mathrm{A}$] \mbox{pKID-KIX}
  }.
\end{equation}
Here, $k_{\mathrm{D}^+}$ is the pseudo first order encounter rate for unbound
molecules to come into proximity through diffusion to form the loosely
associated encounter complex (pKID:KIX). Interacting protein molecules within
the encounter complex either escape with a rate constant $k_{D^{-}}$ or proceed
to form the final bound complex (pKID-KIX) with rate $k_\mathrm{A}$.

For simplicity, the overall binding kinetics described by
Eq.~\ref{eq:kinetic_equation} does not resolve the transitions metastable states
within the encounter complex shown in Fig.~\ref{fig:figure3}.  Following
Ref.~\cite{huang:09},%\onlinecite{huang:09},
the spatial separation between pKID and KIX and
their intermolecular native contacts distinguish the encounter complex from the
unbound and bound states. Kinetic trajectories are started from an unbound
conformation at a separation distance $d_\mathrm{cm} > 40$\AA.  The protein
diffuses until it enters the encounter complex by making at least one native
intermolecular contact. An exit from the encounter complex occurs when all
intermolecular contacts are lost and the center of mass separation distance
between the molecules exceeds 40\AA (becoming unbound), or when the fraction of
native intermolecular contacts exceeds $Q_\mathrm{int} > 0.8$ (becoming bound).
Qualitative kinetic scenarios are not very sensitive to the precise definition
of the encounter complex.

The rate constants in Eq.\ref{eq:kinetic_equation} are calculated from
simulations of independent binding events. A typical simulation consists of
numerous diffusive encounters with the mean time between successive encounters
given by mean first passage time to form the encounter complex,
$\tau_{\mathrm{D}^+}$, plus the mean lifetime of an encounter,
$\tau_\mathrm{ec}$.  Thus, the overall mean binding time, $\tau_{\mathrm{on}}$,
can be expressed as
\begin{equation}
  \tau_\mathrm{on}= \langle n_\mathrm{enc}\rangle \left(\tau_{\mathrm{D}}^{+}+ \tau_\mathrm{ec}\right),
  \label{eq:tau_on1}
\end{equation}
where$\langle n_\mathrm{enc}\rangle$ is the average number of encounters per
binding event. Eq.~\ref{eq:tau_on1} can be written as
\begin{equation}
  \tau_\mathrm{on} = \langle n_\mathrm{enc}\rangle \tau_{\mathrm{D}^+}  + \tau_\mathrm{A},
  \label{eq:tau_on2}
\end{equation}
where $\tau_\mathrm{A} = \langle n_\mathrm{enc}\rangle \tau_{\mathrm{ec}}$ is
the mean time spent in the encounter complex per binding event. The simulated
binding rate, $k_\mathrm{on} = 1/\tau_{\mathrm{on}}$, can therefore be written
as
\begin{equation}
  \label{eq:kon_1}
  \frac{1}{k_{\mathrm{on}}} = \frac{\langle n_{\mathrm{enc}} \rangle}{k_{\mathrm{D}^+}} + \frac{1}{k_{\mathrm{A}}},
\end{equation}
where $k_{\mathrm{D}^+} = 1/\tau_{\mathrm{D}^+}$ and $k_\mathrm{A} = 1/\tau_{\mathrm{A}}$.

Eq.~\ref{eq:kon_1} is a convenient form to analyze simulated binding rates. Nevertheless,
it can be put into somewhat more familiar form by relating the number of encounters
that occur for a typical binding event to the relative rates of exiting the encounter complex
though dissociation, $k_{\mathrm{D}^-}$, or binding, $k_{\mathrm{A}}$.
The dissociation rate can be calculated from the 
mean time spent in the encounter complex before dissociating, $\tau_{\mathrm{D}^-}$. 
If the complex dissociates on average $\langle n_{\mathrm{dis}}\rangle$ times per binding event,
$\tau_{\mathrm{D}^-}$ is given by
\begin{equation}
  \label{eq:tau_dminus}
  \tau_{\mathrm{D}^-} = \tau_{\mathrm{ec}}\frac{\langle n_\mathrm{enc}\rangle}{\langle n_\mathrm{dis}\rangle}
  = \tau_{\mathrm{A}}/\langle n_\mathrm{dis}\rangle.
\end{equation}
Since $\langle n_{\mathrm{enc}}\rangle = \langle n_{\mathrm{dis}}\rangle + 1$,
the dissociation rate, $k_{\mathrm{D}^-} = 1/\tau_{\mathrm{D}^-}$, is related to
$k_\mathrm{A}$ by
\begin{equation}
  \label{eq:n_enc}
  k_{\mathrm{D}^-} = \left( \langle n_{\mathrm{enc}} \rangle - 1 \right) k_{\mathrm{A}}.
\end{equation}

Eq.~\ref{eq:n_enc} reflects how the simulated number of encounters of a typical binding event
is reflected in the relative magnitude of $k_\mathrm{D}^{-}$ and $k_\mathrm{A}$: if
the binding is diffusion limited  ($k_{\mathrm{A}} \gg k_{\mathrm{D}^-}$), every
encounter resulting in a bound complex, $\langle n_{\mathrm{enc}}\rangle \sim 1$; alternatively,
if binding is reaction limited ($k_{\mathrm{A}} \ll k_{\mathrm{D}^-}$), the
encounter complex dissociates many times before binding,
$\langle n_{\mathrm{enc}}\rangle \gg 1$.

Eliminating $\langle n_\mathrm{enc}\rangle$ from Eq.~\ref{eq:kon_1} gives the
simulated mean first passage estimate for the overall rate,
\begin{equation}
  \label{eq:kon_2}
  k_\mathrm{on} =
  \frac{k_\mathrm{D}^{+} k_\mathrm{A}}{k_\mathrm{D}^{+}+ k_\mathrm{D}^{-} + k_\mathrm{A}},
\end{equation}
where the rates on the right hand side are determined from $\tau_{D}^{+}$,
$\tau_{\mathrm{ec}}$, and $\langle n_{\mathrm{enc}}\rangle$ calculated from the
simulations.  As shown in Table~\ref{tab:kinetics}, $k_\mathrm{A}$ for these
simulations are at least an order of magnitude smaller than $k_{\mathrm{D}^+}$
and $k_\mathrm{D}^{-}$ (consistent with many encounters per binding event)
giving the approximate overall binding rate
\begin{equation}
  \label{eq:kon_2}
  k_\mathrm{on} \approx
  \frac{k_\mathrm{D}^{+} k_\mathrm{A}}{k_\mathrm{D}^{+}+ k_\mathrm{D}^{-}}.
\end{equation}


As shown in Table~\ref{tab:kinetics}, coupled folding and binding leads to a
faster rate of entering and a slower rate of escaping the encounter complex than
when pKID is already folded. This agrees qualitatively with the picture
envisioned by fly-casting in which the expanded conformations of an unfolded
protein encourage faster encounter rates, and weak transient attractive
interactions stabilize the encounter complex that result in decreased escape
rate. Nevertheless, the difference in $k_\mathrm{D}^{+}$ and $k_\mathrm{D}^{-}$
are marginal compared to the more than 6-fold decrease in $k_\mathrm{A}$
consistent with a higher free energy barrier of the rigid protein
(Fig.~\ref{fig:figure2}). Binding of both proteins is reaction limited, as seen
from the large number of encounters before succeeding in binding. Since
$k_\mathrm{D}^{-}$ for the rigid pKID is only about 10\% larger than for the
flexible pKID, the increase in $\langle n_\mathrm{enc}\rangle$ for a rigid
protein is primarily due to the decrease of $k_\mathrm{A}$.  These results are
in qualitative agreement with Ref.~\cite{huang:09} which reports that the
primary difference between binding kinetics of flexible and rigid pKID is in the
activated binding step. Without desolvation barriers, the binding is more
diffusion-limited (with order of magnitude smaller
$\langle n_{\mathrm{enc}}\rangle$).

Interestingly, the calculated mean lifetime of an encounter,
$\tau_{\mathrm{ec}}$, is approximately independent of whether pKID is
unstructured or folded prior to binding. This somewhat surprising result is
likely due to the model's neglect of non-native and electrostatic interactions
which likely prolongs the lifetime of the encounter complex of a unfolded
protein more than a typical encounter complex of two folded
proteins.\cite{huang:10a,kim:18d}.

%
%*************************************************************************************************
\section{\textbf{Fly-casting and reaction limited bindings}}\label{sect:three_four}
%*************************************************************************************************
%
Rationalizations for enhanced rates of IDPs usually implicitly assume diffusion
limited bimolecular association. Here, we aim to investigate how the picture of
fly-casting carries over to the case of reaction controlled
kinetics. Desolvation barriers in the contact potential of this model makes it
more likely that successful binding event requires multiple attempts before a
tightly bound complex is formed.

The evolution of local intermolecular order parameters as a function of
separation distance complements description of the binding mechanism shown in
Fig.~\ref{fig:figure3}. Since binding initiates with $\alpha_\mathrm{B}$, we use
the distance between the center of mass of KIX and the center of mass of
$\alpha_\mathrm{B}$, $D_{\mathrm{CM-B}}$, to characterize the proximity of pKID
and KIX.  The separation distance of $\alpha_\mathrm{B}$ in the bound complex
(denoted by $D_\mathrm{nat}$) is approximately 12 \AA.  As shown in
Fig.~\ref{fig:figure4}, weak and diverse interactions between pKID and KIX
initiate when $D_\mathrm{CM-B} \approx 21$ \AA, which defines the outer range of
typical encounter distance, $D_\mathrm{enc}$. 20\% native intermolecular
contacts occur within the C-terminal region of $\alpha_{\mathrm{B}}$, starting
at a closer distance, $D_{\mathrm{CM-B}}$=18 \AA.  Contacts with the N-terminal
end of $\alpha_\mathrm{B}$ begin to develop when pKID and KIX are close within
$D_{\mathrm{CM-B}}$=16\AA, with full binding of $\alpha_\mathrm{B}$ occurring
within 2 \AA~of $D_\mathrm{nat}$,
$14 \mbox{\AA} > D_{\mathrm{CM-B}}> 12\mbox{\AA}$.  Binding of
$\alpha_\mathrm{A}$, which occurs after binding of $\alpha_\mathrm{B}$, is not
well represented by this plot because $D_\mathrm{CM-B}$ does not couple closely
to the binding of $\alpha_{\mathrm{A}}$.

\begin{figure}[htb]
  \centering
  \includegraphics[width=5.5in]{figure4.pdf}
  \caption{Native local contact order $q_{i}$ between pKID and KIX residues shown
    as a function of $\alpha_{\mathrm{B}}$ center of mass separation from
    KIX(b). Normalized $q_{i}$ for pKID residues are represented via color
    density. White dash lines show upper limit of $D_{\mathrm{CM-B}}$ for the
    encounter state denoted as $D_{\mathrm{enc}}$ and its value at native bound
    state $D_{\mathrm{nat}}$.}
  \label{fig:figure4}
\end{figure}

The gradual formation of intermolecular contacts of $\alpha_{\mathrm{B}}$ is
also reflected in the development of the global native contacts within
$\alpha_\mathrm{B}$, $Q_\mathrm{B}$, as a function of separation distance,
$D_\mathrm{CM-B}$.  As shown Fig.~\ref{fig:figure5} (a), nearly 15\% of native
intermolecular contacts of $\alpha_{\mathrm{B}}$ develop at a separation
distance of approximately equal to 21 \AA, a distance that exceeds the native
separation distance, $D_\mathrm{nat}$, by the radius of gyration of
$\alpha_{\mathrm{B}}$ ($R_\mathrm{G}(B)= 7.2$\AA). Incremental progression of
contacts after initial weak interactions occurring at a relatively far distance
on the order of the radius of gyration is in harmony with fly-casting mechanism.
\cite{shoeportman:00}. When $\alpha_\mathrm{B}$ is already structured throughout
the binding, in contrast, intermolecular contacts form simultaneously over a
very narrow range in distance starting at 14 \AA, only 2 \AA~ beyond the
separation in the native state $D_\mathrm{nat}$.

\begin{figure}[htb]
  \centering
  \includegraphics[width=6.0in]{figure5.pdf}
  \caption{ (a) fraction of intermolecular native contacts of
    $\alpha_{\mathrm{B}}$, $Q_B$, and (b) alignment of the end-to-end vector for
    the residues of $\alpha_{\mathrm{B}}$ with the orientation in the bound
    complex, $\cos(\theta_\mathrm{B})$ plotted as a function of center of mass
    separation between $\alpha_{\mathrm{B}}$ and KIX, where red and blue colors
    represent unfolded and folded pKID, respectively.}
  \label{fig:figure5}
\end{figure}

Coupled folding and binding also influences how pKID overcomes orientational
constraints associated with the bound complex. To quantify this, we characterize
the alignment of $\alpha_{\mathrm{B}}$ through the measure,
$\langle\cos\theta_\mathrm{B} \rangle = \langle
\hat{\mathbf{r}}\cdot\hat{\mathbf{r}}_0 \rangle$, where $\hat{\mathbf{r}}$ is
the direction of the relative vector defined by the positions of the terminal
residues of $\alpha_\mathrm{B}$, and $\hat{\mathbf{r}}_0$ is its orientation in
the bound structure.  At far distances, $\alpha_{\mathrm{B}}$ has random
orientations with $\langle\cos\theta_\mathrm{B}\rangle \approx 0$, and in the
bound complex $\alpha_{\mathrm{B}}$ is aligned
$\langle\cos\theta_\mathrm{B}\rangle \approx 1$.  As shown in
Fig.~\ref{fig:figure5} (b), one sees that alignment of $\alpha_{\mathrm{B}}$
tracks development of intermolecular contacts. Here, $\alpha_{\mathrm{B}}$
aligns to the bound orientation incrementally when folding and binding are
concomitant, whereas the entropic loss due to orientation occurs suddenly when
the helix is rigid.  The alignment of the $\alpha_{\mathrm{B}}$ is reminiscent
of the way electrostatic steering enhances binding rates of folded proteins by
increasing the likelihood that a diffusive encounter has the correct orientation
for binding.\cite{huangsteering:15,antosiewicz:96,tan:93} In coupled folding and
binding, the ``long-range'' interactions that orient the helix develop along
with the formation of intermolecular contacts begin over a distance on the order
of the radius of gyration from the separation in the native complex,
$D_\mathrm{CM-B} \approx D_\mathrm{nat} + R_\mathrm{G}(B)$.

Because the alignment develops within the encounter complex, a diffusive
encounter is more likely to successfully bind when folding and binding are
concomitant.  The distribution of $\cos\theta_\mathrm{B}$ for initial encounters
of successful binding attempts illustrates this kinetic difference.  As shown in
Fig.~\ref{fig:figure6}, binding is independent of the alignment of an initial
encounter when folding and binding are coupled. In contrast, successful binding
of a previously folded pKID molecule is much more likely to occur when it
encounters KIX in an orientation similar to that of the bound complex.

\begin{figure}[ht!]
  \centering
  \includegraphics[width=6.0in]{figure6.pdf}
  \caption{Alignment of both $\alpha_{\mathrm{A}}$(red) and $\alpha_{\mathrm{B}}$(blue) are calculated 
    upon successful encounter with target protein KIX illustrated via histogram
    for unfolded(a) and folded(b) ligand pKID. As depicted, both helices show an
    encounter that eventually leads to bound complex(successful binding). Similarly,
    it can be defined as last attempt prior to binding.}
  %\label{fig:align_dist}
  \label{fig:figure6}
\end{figure}


%*************************************************************************************************
\section{\textbf{Mediation of desolvation barriers}}\label{sect:three_five}
%*************************************************************************************************
%
Coupled folding and binding of pKID leads to faster binding kinetics with a
lower free energy barrier separating the unbound and bound conformational
states.  One rationalization for a reduced barrier is through an entropic
stabilization of the encounter complex associated with greater conformational
flexibility, including less restrictive orientational constraints and dynamic
intermolecular interactions.  The way an unstructured protein can use its
conformational flexibility to overcome desolvation barriers is another factor
contributing to faster binding kinetics.  Correlations of individual
intermolecular contacts provide a direct measure of how pKID navigates
desolvation barriers. We consider the correlation of local intermolecular
contacts involving residues $i$ and $j$ of pKID,
\begin{equation}
  c_{ij} = \frac{\left\langle \delta q_i\delta q_j\right\rangle_1}
  %{\sqrt{\left\langle \delta q_i^2\right\rangle_1} \sqrt{\left\langle \delta q_j^2\right\rangle_1}},
  {\sqrt{\langle \delta q_{i}^{2}\rangle_1} \sqrt{\langle \delta{q}_j^2\rangle_1}},
  \label{eq:correlation}
\end{equation}
where  $\delta q_i = q_i - \langle q_i\rangle_1$,  and the averages are over
conformations that have at least one intermolecular contact.

\begin{figure}[ht!]
  \centering
  \includegraphics[width=4.0in]{figure7.pdf}
  \caption{Correlations of intermolecular contacts are depicted 
    for flexible pKID (lower triangle) and folded pKID (upper triangle).
    The color indicates the value of $c_{ij}$. pKID residues that don't make any native
    intermolecular contact are omitted.}
  \label{fig:figure7}
  %\label{fig:corr_plot}
\end{figure}

Intermolecular contact correlations for each residue are shown in
Fig.~\ref{fig:figure7}.  By definition, $c_{ij}$ is normalized to 1 for
correlations along the diagonal ($i=j$). Moving away from a point on the
diagonal corresponding to $c_{ii}$ in the horizontal and vertical direction
indicates how intermolecular contacts of other residues correlate with
intermolecular contact of the $i^{\mathrm{th}}$ residue.  As shown in
Fig.~\ref{fig:figure7}, when folding and binding are coupled, the correlations
of intermolecular contacts with residues of $\alpha_\mathrm{B}$ decrease as the
sequence separation increases. In contrast, intermolecular contacts with
residues of $\alpha_\mathrm{A}$, which is more structured and binds later than
$\alpha_\mathrm{B}$, are highly correlated throughout the helix.  That is,
contacts between KIX and residues in $\alpha_\mathrm{B}$ are made and broken
relatively independently, while those of $\alpha_{\mathrm{A}}$ occur
collectively.

In quantifying the characteristic sequence length describing these correlations,
we focus on the contacts of $\alpha_\mathrm{B}$ because binding initiates with
this helix.  For residues in $\alpha_\mathrm{B}$, the average correlation falls
off exponentially with sequence separation,
$\overline{c_{|i-j|}} = \exp(-|i-j|/l_0)$ with $\l_0 = 2$, as shown in
Fig.~\ref{fig:figure8}. That is, contacts in $\alpha_\mathrm{B}$ tend to occur
with regions containing roughly 5 residues. In contrast, when pKID is structured
throughout binding, the contact correlations behave as
$\overline{c_{|i-j|}} \approx 1 - |i-j|/l_0$ with $l_0 = 50$, about five times
longer than the length of the helix. Accordingly, when $\alpha_{\mathrm{B}}$ is
already folded, the residues form intermolecular contacts collectively over the
entire helix.

\begin{figure}[ht!]
  \centering
  \includegraphics[width=4.0in]{figure8.pdf}
  \caption{Average correlations as a function of sequence separation of
    $\alpha_{\mathrm{B}}$ for flexible pKID(red) and folded pKID(red) is
    shown. A fit(gray line) to these correlations give decay lengths of
    $l_0 \approx 2$ for flexible and $l_0 \approx 50$ for rigid pKID
    respectively.}
  %\label{fig:corr_fit}
  \label{fig:figure8}
\end{figure}


The ability of a flexible, unfolded protein to navigate desolvation barriers
with relatively small local steps results in lower kinetic barrier within the
encounter complex. Structural constraints associated with folded proteins, in
contrast, make simultaneous desolvation over a broader area more likely, which
leads to a high free energy barriers and slower kinetics. This picture is in
harmony with the coupled folding and binding gradual accumulation of
intermolecular contacts as the proteins approach each other shown in
Fig.~\ref{fig:figure5}.


%
%*************************************************************************************************
\section{\textbf{Binding affinity, binding rates, and unbinding rates}}\label{sect:three_six}
%*************************************************************************************************
%
Folding upon binding allow IDP to form highly specific, yet low affinity bound
complexes\cite{dyson:05,hsu:13ex,tompa:08f}.  Thermodynamically, the lower
affinity can be understood by entropic stabilization of the unbound disordered
protein due to its higher conformational freedom. From a kinetic point of view,
the binding affinity, $K_\mathrm{a}$ is determined by the bimolecular binding
rate, $k_\mathrm{bi}$, and the unimolecular unbinding rate, $k_\mathrm{off}$,
through
\begin{equation}
  \label{eq:Ka}
  K_\mathrm{a} = \frac{k_\mathrm{bi}}{k_\mathrm{off}}.
\end{equation}
So far we have considered the influence of coupled folding and binding on
$k_\mathrm{bi}$ (or more specifically, $k_\mathrm{on} = k_\mathrm{bi}[L]$ where
$[L]$ is the concentration of pKID for the simulation). In this section, we
investigate how changes in the off rate in addition to the binding rate lead to
lower affinity complexes though coupled folding and binding.

Although the $k_\mathrm{bi}$ is enhanced through coupled folding and binding
(which tends to increase the affinity), $k_\mathrm{off}$ must have a greater
increase consistent with the IDP's low binding affinity. As emphasized by Zhou,
while there has been more attention to the influence of coupled folding and
binding on $k_\mathrm{bi}$, its influence on $K_\mathrm{a}$ is restricted by an
upper diffusion limited rate. The dissociation rate, which does not have such a
limitation, can be adjusted by coupled folding binding to be fast enough for
regulatory functions.\cite{zhou:12}

The bound complex pKID-KIX is much less stable when folding and binding are
coupled as indicated by its low binding midpoint temperature ($T_0=321^\circ$ K)
compared to that the midpoint temperature when pKID is folded before binding
($T_0=517^\circ$ K).  Since it is difficult to reliably estimate the bound
population from simulations at temperatures far from $T_0$, we reduce binding
affinity of rigid to bring it closer to affinity of the flexible pKID by
weakening the intermolecular energy parameter for the folded pKID from
$\epsilon_{\mathrm{int}}=2.1\epsilon_{ij}$ to
$\epsilon_{\mathrm{int}}=1.3\epsilon_{ij}$ to get a qualitative understanding of
how coupled folding and binding influences the binding kinetics and affinity.
The resulting equilibrium constants and unbound populations as a function of
temperature are shown in Fig.~\ref{fig:figure9}.  Without this reduction in
binding affinity through $\epsilon_{\mathrm{int}}$, the difference in binding
affinities is estimated to be as high as four to six orders of magnitude at
$T_0=321^\circ$K.

\begin{figure}[h!]
  \centering
  \includegraphics[width=6.0in]{figure9.pdf}
  \caption{Dissociation constant(a) and unbound population (b) as a function of
    temperature for binding when pKID is unstructured (red) and folded (blue).
    Vertical dash lines are located at corresponding melting temperatures of
    flexible and rigid.}
  \label{fig:figure10}
\end{figure}

As shown in Table~\ref{tab:affinity_kinetics}, with the adjustment in
parameterization, coupled folding and binding reduces $K_\mathrm{a}$ by a factor
of 270. This reduction results from an increase of $k_\mathrm{bi}$ by a factor
of 5, and a much greater increase of $k_\mathrm{off}$ by nearly a factor of
2000.  These results are consistent with the proposal that the change in the
affinity due to coupled folding and binding is primarily reflected in greatly
enhanced off rate.\cite{zhou:12}


\begin{table}[t!]
  \centering
  \begin{tabular}{@{}llllllll@{}}
    & $k_{\mathrm{on}}^{o}$ \footnote{On/off rates are expressed in $10^{-10} [\Delta t^{-1}]$.}
    & $k_{\mathrm{bi}}$  \footnote{Bimolecular rate is expressed in $10^{-10} [\Delta t^{-1}\mathrm{\mu M^{-1}}]$.}
    & $k_{\mathrm{off}}$
    & $K_{\mathrm{d}}$
    & $\mathrm{P_{B}}$ \\
    %  & $\epsilon_{\mathrm{int}}$\\
    \hline
    flexible  &  3800  & 3.0  & 2200  & 736   &  0.57  \\%&  2.1$\epsilon_{0}$ \\
    rigid     &  540   & 0.4  & 1.1   & 2.7   &  0.96  \\%&  1.3$\epsilon_{0}$ \\
  \end{tabular}
  \caption{ These results are calculated from simulations at $T=321 ^\circ$K, where
    concentration units are in $\mu \mathrm{M}$. }
  \label{tab:affinity_kinetics}
\end{table}


%The molecular dissociation mechanism of incremental breaking and reforming
%contacts allowed when the protein unfolds when it dissociates is similar to the
%binding mechanism, though it is easier to see directly in the simulations in
%individual unbinding events.
The molecular dissociation mechanism of incremental breaking contacts allowed when
the protein unfolds. The mechanism of dissociation is similar to binding
mechanism, though it is easier to see directly in the simulations in individual
unbinding events.
As shown in Fig.~\ref{fig:figure10} the free
energy barrier controlling the dissociation rate is lower when pKID unfolds upon
dissociation. Segments of typical unbinding trajectories are also shown in
Fig.~\ref{fig:figure10}. Here, we see that changes in the intermolecular
contacts occur with small incremental changes, often reforming and breaking
before overcoming the dissociation barrier. In contrast, when pKID is folded
after dissociation, structural constraints require that contacts are lost
collectively requiring it to overcome a larger collective desolvation barrier.


\begin{figure}[h!]
  \centering
  \includegraphics[width=6.0in]{figure10.pdf}
  \caption{Visual illustration of intermolecular contact formation for
    $\alpha_{\mathrm{B}}$ as it approaches its binding pocket is presented via
    2D free energy plots for flexible(a) and rigid(b) pKID.  Center of mass
    distance between $\alpha_{\mathrm{B}}$ and KIX is used to characterize their
    separation. In this, the most representative unbinding trajectory is shown
    on top of free energy landscape.}
  \label{fig:figure9}
\end{figure}

%
%*************************************************************************************************
\section{\textbf{Materials and Methods}}\label{sect:three_seven}
%*************************************************************************************************
%
Coarse-grained, native-centric Go model\cite{okazaki:06} simulations are
performed with $\textit{Cafemol}$\cite{Kenzaki:11} software. Role of solvent is
implicitly incorporated via Langevin dynamics with $\Delta \mathrm{t}$=0.2
Velocity Verlet integration time step and $\gamma$=0.25 friction coefficient in
$\textit{Cafemol}$ units. AICG1 model\cite{li:11} is employed to account for
sequence specificity of both local and non-local interactions.

For simplicity, target protein (KIX) is constrained to its native structure,
while pKID is free to move inside a cubic box with non-periodic boundary
conditions. For equilibrium simulations the box has dimensions
90\AA. Simulations for binding rate calculations are carried in a 110\AA~cubic
box with independent 200 binding and unbinding runs. Binding simulations are
started at random unbound and distant confirmations and are stopped once bound
complex is formed, wheres dissociation simulations are initiated at native state
and finalized once all intermolecular contacts are lost.

Native contacts between pairs of amino acids are defined using all atom
information of the reference structure, where if the distance between any
non-hydrogen atoms of different pairs of residues is less than $6.5$\AA~ then
they are considered to be in physical contact with each other.  Structural
similarity of a conformation with the native structure is characterized by the
fraction of native contacts, where folding and binding are represented through
fraction of native intramolecular($Q_{\mathrm{intra}}$) and
intermolecular($Q_{\mathrm{inter}}$)contacts respectively.  For analysis, native
intramolecular contact is considered formed when
$r^{ij} \leq 1.2r_{\mathrm{nat}}^{ij}$, where $r^{ij}$ is separation distance
between pairs of residues and $r_{\mathrm{nat}}^{ij}$ separation distance at
native state.

Intermolecular contacts interact through a pair potential which includes a
desolvation barrier separating a weakly stabilizing solvent separated minimum
potential and the full strength contact at closer distances. The form of this
potential is described in Ref.~\cite{cheung:02} is parameterized with a
desolvation barrier height $\epsilon_{\mathrm{db}}=0.1\epsilon_{ij}$ at
$r_\mathrm{db} = r_{\mathrm{nat}} + r_{\mathrm{water}}$ and solvent separation
minima depth $\epsilon_{\mathrm{ssm}}=\epsilon_{ij}/3$ at
$r_\mathrm{ssm} = r_{\mathrm{nat}} + 2r_{\mathrm{water}} $, where
$\epsilon_{ij}$ is sequence specific pairwise energy parameter. For analysis,
contacts are considered formed when $r^{ij} \leq r_{\mathrm{db}}$
($r_{\mathrm{water}}$=1.5\AA).

Weighted Histogram Analysis Method\cite{kumarWHAM:92}(WHAM) is used to combine
long equilibrium runs simulated at different temperatures, which enables us to
get statistical average of any order parameter at any intermediate
temperature\cite{noel:10}. The dissociation constant is estimated by
\begin{equation}
  \label{eq:kd}
  \mathrm{K_{d}}= [C]\frac{P_{\mathrm{U}}^2}{P_{B}},
\end{equation}
where $[C]$ is protein concentration, $P_{\mathrm{B}}$ is the bound population,
and $P_\mathrm{U} = 1 - P_\mathrm{B}$ is the unbound population. The bound
population at a temperature $T$ is calculated by integrating of the potential of
mean force
\begin{equation}
  \label{eq:pu}
  \mathrm{P_{\mathrm{B}}}=\int_{Q^*}^{1} e^{-\beta F(Q)}dQ/\int_{0}^{1} e^{-\beta F(Q)}dQ,
\end{equation}
where $\beta = 1/k_\mathrm{B} T$.  Here, $Q^*=0.5$ defines intermolecular
contact threshold that distinguishes the bound and unbound populations.

In order to match experimental binding affinity\cite{zor:02} of the IDP complex
we increased intermolecular energy prefactor until desired affinity
$\mathrm{K_{a}}=0.33 \mu \mathrm{M^{-1}}$ is achieved at
$\epsilon_{\mathrm{int}}=2.1\epsilon_{\mathrm{ij}}$ and
$\mathrm{T}=303\mathrm{K}$.  Above conditions result in highly stabilized bound
state with $P_{\mathrm{B}}=0.96$ that makes unbinding simulations too slow to
simulate directly in a reasonable time. We overcome this problem by estimating
off rate using Eq.~\ref{eq:Ka}.

Similarly, using above intermolecular energy prefactor
$\epsilon_{\mathrm{int}}=2.1\epsilon_{\mathrm{ij}}$ for folded pKID result in
very tight bound complex with high melting temperature
$\mathrm{T_{0}}=517\mathrm{K}$.  We reduce this energy parameter to
$\epsilon_{\mathrm{int}}=1.3\epsilon_{ij}$ in order to make comparison of
binding affinities between flexible and rigid pKID practical (Fig.~\ref{fig:figure11}).

\begin{figure}[h!]
  \centering
  \includegraphics[width=4.0in]{figure11.pdf}
  \caption{Heat capacity vs temperature plots for binding/unbinding simulations
    of complex are prepared thermodynamic sampling for both flexible(red) and
    rigid(blue) binding partners.}
  \label{fig:figure11}
\end{figure}


Experiments suggest that two helices of pKID do not show the same level of disorder,
where helicity of $\alpha_{\mathrm{A}}$ is $50\%-60\%$ , and for
$\alpha_{\mathrm{B}}$ it is $10\%-15\%$ \cite{radhakrishnan:98}. To model this,
we tuned local(bond angle and dihedral) and non-local(Go) energy parameters for
both $\alpha_{\mathrm{A}}$ and $\alpha_{\mathrm{B}}$. $\alpha_{\mathrm{A}}$
rigidity is relaxed by reducing its energy parameter twice from original value
that gave $\langle q_{\mathrm{a}}^{\mathrm{free}}\rangle=0.54$ and kept the same
throughout. For $\alpha_{\mathrm{B}}$, weakening the energy parameter 20 times
resulted in $\langle q_{\mathrm{b}}^{\mathrm{free}}\rangle = 0.32$.  To explore
the influence that chain flexibility has on the binding kinetics, we also
consider a hypothetical rigid pKID protein in which the energy parameters of
$\alpha_{\mathrm{B}}$ are kept at their original values.

%*************************************************************************************************
\section{\textbf{Concluding Remarks}}\label{sect:three_nine}
%*************************************************************************************************
In this paper, we present simulations of a structure based model with
desolvation barriers separating unbound and bound intermolecular interactions
for the coupled folding and binding of pKID-KIX.  Like previous work, the
binding mechanism of pKID-KIX complex is in qualitative agreement with the
experiment\cite{sugase:07}. Desolvation increases the barrier between bound and
unbound states allowing us to assess how flexibility and coupled folding and
binding influence the binding mechanism outside of the diffusion limited
regime.  Perhaps surprisingly, we find that coupled folding and binding enhances
binding rates to a greater extent than when binding is more diffusion limited.
\cite{huang:09,huang:10a,shoeportman:00,turjanski:08,levy:04} 
The enhanced rates encouraged by coupled folding and binding in the reaction limited regime
is rationalized by fly-casting along with an unstructured protein's ability
to navigate intramolecular desolvation barriers incrementally because of its increased flexibility.

Our study also emphasizes how flexibility tunes binding, unbinding, and binding
affinity.  By comparing kinetics of flexible and rigid pKID at the same
temperature, we find that while the binding rate is enhanced, the unbinding rate
is accelerated to a much greater extend, resulting in a much less stable bound
complex. Rapid unbinding and low affinity complexes have been argued to be more
essential to IDP function than rapid binding.\cite{zhou:12} The simulations
described in this paper suggest IDPs dissociate rapidly because they can break
contacts in small independent steps, while a rigid protein must break them
simultaneously.  Unbinding rates of IDPs have been emphasized in another recent
study using coarse-grained models.  Umezawa \textit{et al.}, for example, showed
that disorder induced rate enhancement of pKID-KIX is greater for dissociation
rate than the association rate, although they compared models with much closer
affinities than in the current paper\cite{umezawa:16}. In addition the
dissociation rate as well as affinity has also been used to tune interaction
parameters of structure based models of IDPs.\cite{cao:16}
Our results further emphasize the importance of IDPs as regulatory and signaling proteins that
achieve rapid molecular recognition with high specificity and low affinity facilitated by
their intrinsic conformational flexibility.

\end{document}
