\documentclass[../talant.diss.submit.tex]{subfiles}
\begin{document}
\label{chap:outlook}
%Functions performed by IDPs in a cell are complementary to those of
%ordered proteins\cite{}.
The conformational flexibility of IDPs give them particular
%Because of their conformational flexibility, IDPs posess
functional advantages over ordered proteins.
\cite{dyson:16m,berlow:15f,zhou:12,dunker:01int,dyson:05,
  shoeportman:00,uversky:05,gunasekaran:03,kriwacki:96,liu:14a}
%\todo{can either not
%  cite or cite more and more recent reviews.}
The main goal of the research
presented in my dissertation is to study some of these advantages using
structure based molecular dynamics simulations.  Particular properties I
consider in my research are enhanced association rates, less restrictive
orientational binding constraints, low affinity and high specificity binding,
and the influence that the speed of conformational dynamics has on binding rates
and mechanisms.  I address these aspects of IDP binding to its target protein
with some modifications implemented into coarse grained C-$\alpha$ model using
well characterized pKID-KIX IDP complex as a model system. In this
chapter, I offer a brief summary of my results in broad terms,
and describe some open questions I
find promising for future research.

In a first project, our properly parametrized IDP model predicts three step
binding mechanism in qualitative agreement with experimental
observation.\cite{sugase:07} Inclusion of a desolvation barrier potential for
native contacts resulted in a significantly enhanced coupled folding and binding
rate over binding of a prefolded molecule.  This study is conceptually
interesting because it illustrates the influence conformational flexibility
has on the binding rate and mechanism  when binding is reaction limited.
The enhanced rates are rationalized by the ability of flexible binding partner
to navigate the desolvation barriers incrementally starting at a farther
separation distance, while an ordered counterpart must overcome these barriers
simultaneously due to structural constraints. In addition, I show how binding
affinity and the specificity are uncoupled by comparing flexible and rigid pKID
binding kinetics at the same temperature.  These results suggest that structure
based C-$\alpha$ model is robust in accurately capturing fine detail of
experimental observations.

One interesting open problem for which I think our model is well suited is the 
% Being moivated by this work, I aim to pursue an
investigation of the role of phosphorylation on pKID-KIX binding. % with C-$\alpha$ model. 
Phosphorylation is a reversible mechanism that consists of the addition of a
phosphate group ($\mathrm{PO}_{4}$) to the polar group R of various amino acids
allowing the protein to change conformation when interacting with other molecules.
For an IDP pKID the phosphorylation of a specific amino acid SER133 is
known to increase its binding affinity to KIX\cite{radhakrishnan:98,kwok:94n},
although it is unclear what mechanism governs this process. In C-$\alpha$ model,
the effect of phosphorylation can be implemented by altering the residual structure, binding
strength and electrostatic energy parameters of specific region. Particularly,
more extensive analysis must be performed on the formation of transient encounter
complex and its response to the phosphorylation effect as it is suggested to
have great influence.\cite{dahal:17ph}


%textbf{summarize reconfiguration dynamics vs rate, 4rd chap.}\\

In a second project, I study the role of polymeric reconfiguration timescale
on coupled folding and binding kinetics and thermodynamics using C-$\alpha$ model.
This is accomplished by decoupling the langevin dynamics into an independent equations
that enable control over reconfigurational speed and the center of mass diffusion of an
IDP separately. Using an IDP pKID, the implementation is validated by showing that
diffusion constant remains unchanged while internal dynamics parameter $\gamma_{\mathrm{I}}$
is gradually changed. Results show that thermodynamics of binding is not affected by changes
in the internal dynamics, while nearly order of magnitude difference in binding
rate is observed between the fast and slow regimes.
Kinetic signatures of the slower conformational speed is revealed 
% This differance in kinetics is rationalized
by dissecting the binding trajectory into multiple intermediate steps including a loose encounter complex
with a wide distribution of non-native contacts but with minimal native contacts,
and an encounter complex with more extensive native intermolecular contacts. 
My results suggest that the lifetime of the encounter complex is kinetically stabilized,
similar in some respects (but also with important differences) to the thermodynamic
stabilization due to non-native interactions. 
%to mechanism shown with non-native interactions.
%Coformational dynamics of IDPs are found to play an important role in an initial stages of protein
%aggregation\cite{ahmad:12,ahmad:12a} and shown to have an influence on binding mechanism\cite{zhou:12a}.
%Calibration of the internal dynamics
%of an IDP in structure based C-$\alpha$ model, the reconfiguration dynamics can be slowed
%down to be close to experimentally
%reported range. As a result, encounter complex is stabilized due to ...

The model I developed can be used to address another interesting question about
coupled folding and binding proceeding via either induced fit or conformational
selection mechanisms\cite{baker:14,espinoza-fonseca:09a,sugase:07} that is
subject of a debate within the community.  One illustration of this issue is the
so-called gated diffusion limited reaction.
%Characterization of these
%possible mechanisms is made by Zhou \textit{et al.}
In an analytical model of stochastically gated chemical reaction, Zhou
\textit{et al.} show how the binding rate depends on the relative timescale of
gating and the diffusive ligand encounter time.\cite{cai:11} According to this
model, when the rate of conformational dynamics is slow compared to the ligand
encounter time the binding proceeds via conformational selection, whereas fast
conformational dynamics results in a binding rate that is an average of binding
to the closed and open states possibly showing a shift from conformational
selection to induced fit mechanism. Although this analytical model offers a
great control over gating transition rate between open and closed states of the
protein, it lacks an accurate molecular level characterization of the problem.
In contrast, the model we implemented offers a control over proteins
conformational speed with explicit molecular description. If the conformational
speed of an IDP is significantly slow relative to its diffusive encounter
timescale, upon encounter the conformational sampling of an IDP is restricted by
its speed. As a result, the transition from unbound conformation to bound state
conformation is optimized and controlled by weak interactions in the encounter
state leading to a target protein selecting the bound conformation of the ligand
before binding. Thus, I expect to see a shift from induced fit to conformational
selection binding mechanism.  The limitation we have on the maximum limit of
$\gamma_{\mathrm{I}}$ due to integration inaccuracies can be overcome either by
modifying Honeycut-Thurumalai approximation by including higher degree
polynomials or by changing $\gamma_{\mathrm{D}}$ while $\gamma_{\mathrm{I}}$ has
a value close to experimentally known reconfiguration timescale.

%\textbf{Talk about how it can be applied to other systems, or
% possibly with different modifications, and discuss
% improovements it can show giving some experimental references!} \\
%Effect of conformational speed with long range electrostatic interactions..... \\

%\todohl{I uncommented this. I think it is fine. You can take it out again if you like.}
%{
Another problem suggested by this work is to probe
%I am also interested in expanding on this work by probing
the role of long range electrostatic
interactions (as opposed to short range non-native contacts) on IDP binding with slow and fast
conformational speed. Electrostatic interactions are known to speed up binding rate by
orientating a ligand at farther distances, so called "Electrostatic Steering" effect. High
conformational entropy of rapidly rearranging ligand likely make IDP with fast reconfiguration
less sensitive to steering effect. Thus, I predict to see more pronounced acceleration in binding
rate at slow ligand reconfiguration compared to fast one.
%}

IDPs are different from conventional structured proteins in structure, sequence and function.
Conformational dynamics and flexibility allow IDPs perform vital functions in cell that are
complementary to those ordered proteins.
%Binding of IDPs with other targets remains very active
%area and reqires both experimental and computational approaches to desipher.
Molecular recognition of IDPs is a  complex mechanism for which conformational flexibility and
dynamics, specificity, and thermodynamic and kinetic control meet. Deciphering the molecular
level description of this rich mechanism will likely motivate new complementary experimental
and computation methods for years to come. 

\end{document}
